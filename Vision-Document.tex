\documentclass[11pt, a4paper]{supplemental-document}
%my code for color
%\documentclass[11pt,a4paper]{article}

\usepackage[utf8]{inputenc}
\usepackage{graphicx}
\usepackage{microtype}
\usepackage{float}
\usepackage{rotating}
\usepackage{xcolor}
\usepackage{sectsty}
\usepackage{url}
\usepackage{hyperref}
\usepackage{fancyhdr}



\pagestyle{fancy}
\fancyhf{}
\rhead{SOEN 6481\\Summer 2020}
\lhead{Concordia Univeristy\\CS \& SE dept.}
\chead{Vision-Document }
\rfoot{Page \thepage}

\hypersetup{
    colorlinks=true,
    linkcolor=blue,
    filecolor=magenta,      
    urlcolor=cyan,
}

\urlstyle{same}
\definecolor{darkgray}{rgb}{0.66, 0.66, 0.66}
\definecolor{blue(ryb)}{rgb}{0.01, 0.28, 1.0}
\definecolor{dimgray}{rgb}{0.41, 0.41, 0.41}

\sectionfont{\color{black}} 
\subsectionfont{\color{blue}}
%my code for color

\setboolean{shortarticle}{false}

\title{Vision-Document}
\author{} %leave this blank
%% DO NOT ADD AUTHOR INFORMATION HERE; IT WILL BE ADDED DURING PRODUCTION

\begin{abstract}

\end{abstract}

\setboolean{displaycopyright}{false} %copyright statement should not display in the  supplementary document

\begin{document}

\begin{center}
 {\huge Vision-Document\\Project Name}
\end{center}

\section{Introduction}

The purpose of this project was to.......................




%-------------Positioning---------------------
\section{Positioning}
\subsection{Problem Statement}

\begin{table}[H]
\centering
\begin{tabular}{||p{4cm}||p{7cm}||}
\hline

The problem of & {- \newline
- }\\
\hline
Affects & {- \newline
-  }\\
\hline
The impact of which is & {- \newline
- }\\
\hline
A successful solution would be & { }\\
\hline
\end{tabular}
\label{tab:problem statement}
 \end{table}
%----------------------------------------
\subsection{Product position Statement}
\begin{table}[H]
\centering
\begin{tabular}{||p{4cm}||p{7cm}||}
\hline

For & {- \newline
- }\\
\hline
Who & {-\newline
-  }\\
\hline
The Project Name & {- \newline
- }\\
\hline
That & { }\\
\hline
\hline
Unlike & { }\\
\hline
\hline
Our Product & { }\\
\hline
\end{tabular}
\label{tab:problem statement}
 \end{table}
%-------------------------------------------
\section{Stakeholder Descriptions}
\subsection{Stakeholder Summary}
\begin{table}[H]
\centering
\begin{tabular}{||p{4cm}||p{5cm}||p{5cm}||}
\hline
Name & {Description }&{Responsibilities}\\
\hline
\hline
[stakeholder type.] & {[stakeholder description] &   [] }\\
\hline
\hline
\end{tabular}
\label{tab:problem statement}
 \end{table}
%---------------------------------
\subsection{User Environment}
[Detail the working environment of the target user. Here are some suggestions:\\
Number of people involved in completing the task? Is this changing?\\
How long is a task cycle? \\

%-------------------------------------------
\section{Product Overview}
\subsection{Product Perspective}
This subsection of the Vision document puts the product in perspective to other related products\\
\subsection{Assumptions and Dependencies}
List each factor that affects the features stated in the Vision document.\\

\begin{table}[H]
\centering
\begin{tabular}{||p{4cm}||p{7cm}||}
\hline

Assumptions & {Dependencies }\\
\hline
[state any assumptions]	 & { }\\
\hline
\hline
[ ]	 & { }\\
\hline
\hline
[ ]	 & { }\\
\hline
\end{tabular}
\label{tab:problem statement}
 \end{table}
%-----------------------------------
\subsection{Needs and Features}

[Avoid design. Keep feature descriptions at a general level. Focus on capabilities needed and why (not how) they should be implemented.]\\

\begin{table}[H]
\centering
\begin{tabular}{||p{3cm}||p{2cm}||p{2cm}||p{4cm}||}
\hline

Need & Priority & Features & Planned Release\\
\hline
[state a need]	 & {[Set priority:
High, Normal, Low] } & {[Name the feature]	 } & {---- }\\
\hline

\end{tabular}
\label{tab:problem statement}
 \end{table}
%------------------------------------
\subsection{Alternatives and Competition}
	
[Identify alternatives the stakeholder perceives as available. These can include buying a competitor’s product, building a homegrown solution, or simply maintaining the status quo. List any known competitive choices that exist or may become available. Include the major strengths and weaknesses of each competitor as perceived by the stakeholder or end user.]\\

%-------------------------------------------
\section{Other Product Requirements}
[At a high level, list applicable standards, hardware, or platform requirements; performance requirements; and environmental requirements.\\
Define the quality ranges for performance, robustness, fault tolerance, usability, and similar characteristics that are not captured in the Feature Set.\\
Note any design constraints, external constraints, or other dependencies.\\
Define any specific documentation requirements, including user manuals, online help, installation, labeling, and packaging requirements.\\
Define the priority of these other product requirements. Include, if useful, attributes such as stability, benefit, effort, and risk.]\\

%-------------------------------------------
\section{Contribution Table}
\begin{table}[H]
\centering
\caption{\bf Task done by each team member :}
\begin{tabular}{||p{4cm}||p{7cm}||}
\hline
Team Member & Contribution \\
\hline\hline
name1 & {-  \newline
-  \newline
-  \newline
- }\\
\hline
Mridul Pathak & {- \newline
-  \newline
-  \newline
-  }\\
\hline
name3 & {- \newline
- \newline
- }\\
\hline
name4 & {
-  \newline 
-  \newline
-  }\\
\hline
name5 & {
-  \newline
-  \newline
- } \\
\hline
\end{tabular}
\label{tab:Work done by each team member}
 \end{table}
%-----------------Glossary ---------------------------
\section{Glossary}

\begin{table}[H]
\centering
\caption{\bf Glossary :}
\begin{tabular}{||p{2cm}||p{9cm}||}
\hline
Term & Definition \\
\hline\hline
abbreviation1 & full form\\
\hline
\end{tabular}
\label{tab:Work done by each team member}
\end{table}

\newpage
%-------------References ---------------------
\addcontentsline{toc}{section}{References}\begin{thebibliography}{1}

\bibitem{Latex}
 \url{https://www.overleaf.com/learn/latex/Commands}
 Latex documentation and format reference\\
\end{thebibliography}
\end{document}
